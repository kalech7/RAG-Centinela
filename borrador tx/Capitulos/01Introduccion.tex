\chapter{Introducción}

\section{Planteamiento del problema}

\section{Justificación}

\section{Justificación Metodológica}

\section{Objetivos}
\subsection{Objetivo general}
Desarrollar e implementar un sistema RAG que mejore el desempeño del buscador de la plataforma
Centinela, permitiendo recuperar información científica relevante y generar respuestas automáticas
de valor para el usuario.
\subsection{Objetivos específicos}
\begin{itemize}[align=left, label=-]
    \item Realizar una revisión sistemática de la literatura sobre metodologías y/o frameworks para la implementación de RAG.
    \item Diseñar e implementar la arquitectura técnica del sistema RAG utilizando modelos de recuperación y generación de texto.
    \item Evaluar el sistema RAG desarrollado mediante métricas estándar.
\end{itemize}

\section{Alcance}

\section{Marco Teórico}

\section{Revision de literatura}

introduccion poner 

\subsection{Propósito y objetivos de la revisión}
El propósito de esta revisión es consolidar la información disponible sobre los RAG, abordando su estudio desde los fundamentos hasta las fases de desarrollo. 
Se inicia con su definición y arquitectura, para luego profundizar en las etapas clave del proceso: Extraccion del corpus, preprocesamiento, vectorización, recuperación de información, 
evaluación, almacenamiento en bases vectoriales y generación de resultados. Asimismo, se examinan los paradigmas, las métricas de evaluación y el futuro de RAG.
Durante esta revision se busca lograr el objetivo general de proporcionar un panorama global y actualizado sobre los RAG,
exponiendo sus fundamentos, desarollo y aplicación.

\subsection{Criterios de inclusión y exclusión}
Se incluyen únicamente revisiones sistemáticas y metaanálisis publicados entre 2018 y 2025, en inglés o español, dado que la producción 
científica en el área comenzó a incrementarse a partir de 2018, con base en información de Lens.org\footnote{Es una plataforma abierta
para la búsqueda, análisis y visualización de literatura científica y patentes. Accesible en: \href{https://www.lens.org/}{Lens.org}},  Este incremento coincide
con la popularización 
de los modelos de lenguaje basados en transformers.\footnote{Se atribuye a hitos como BERT (2018), GPT-2 (2019) y T5 (2020), que impulsaron un avance 
en la investigación del procesamiento del Lenguaje Natural}
Los estudios deben provenir de fuentes confiables y ser, a su vez, revisados por 
un experto. Se da preferencia a aquellos que presenten una cobertura amplia de los temas más relevantes para el objeto de estudio.  

Se excluyen las revisiones narrativas, los documentos que carezcan de transparencia en sus métodos de búsqueda o síntesis, así como las publicaciones que no estén 
directamente relacionadas con el objeto de estudio delimitado.

\subsection{Identificación del estudio semilla y selección de revisiones relevantes}
El proceso de búsqueda se inicia con la identificación de dos estudios semilla, extraídos de Google Scholar mediante los parámetros “retrieval information” y “retrieval augmented generation”. 
Debido al análisis realizado en Lens.org, se estableció el filtro de 2018 a 2025, ya que se observa que a partir de 2018 el término retrieval-augmented generation 
comenzó a adquirir una relevancia en la literatura científica, mostrando interés de la comunidad investigadora hasta la actualidad.

El primer estudio seleccionado fue Information Retrieval: Recent Advances and Beyond (Hambarde \& Proença, 2023), publicado en IEEE Access. 
Este trabajo constituye una revisión exhaustiva de la recuperación de información, abarcando desde los métodos tradicionales hasta los enfoques 
basados en deep learning y transformers, por lo que resulta un punto de partida principal para explorar la literatura reciente y relevante.

El segundo estudio semilla corresponde al artículo Retrieval-Augmented Generation for Large Language Models (Gao, Xiong, Gao, Jia, Pan, Bi, Dai, Sun \& Wang, 2023), publicado en
arXiv,
el cual presenta un marco conceptual y aplicado sobre la integración de recuperación de información y modelos generativos de gran escala. Su incorporación permite 
establecer una base teórica para contextualizar el análisis de las revisiones seleccionadas.

A partir de estos dos estudios semilla, y aplicando los criterios de inclusión y exclusión previamente definidos, se identificaron \textcolor{blue}{25} revisiones 
relevantes que cumplen con los criterios establecidos. 
Estas revisiones constituyen la base para el análisis y síntesis en el presente trabajo.



\subsection{Valoracion de la evidencias y extracion de la infomacion}
De los estudios seleccionados se procede a realizar un análisis, 
con el fin de excluir aquellos artículos que no cumplen con los criterios establecidos 
o que presentan un nivel de profundidad insuficiente para los objetivos de la revisión. 
La selección final de los estudios se realiza en consenso con expertos en el área, 
garantizando así la pertinencia y relevancia de la evidencia incluida.Para la organización, codificación y síntesis de la 
información se usa ATLAS.ti \footnote{Scientific Software Development GmbH. Disponible en: \href{https://atlasti.com/es}{Atlas.ti}} que facilitará la estructuración de los hallazgos.





% para resuemir la informacion se uso (herramienta) atlas por ejemplo

\subsection{Síntesis y representación de resultados}
Con la literatura seleccionada se identificó la hoja de ruta que se presenta a continuación en la Figura \ref{fig:secciones-rag}.

% --- Grafico resumen

\begin{figure}[H]
\begin{center}
\begin{tikzpicture}[x=1cm,y=1cm]
  % distancia vertical entre pastillas
  \def\step{-1.5}

  % y inicial (arriba)
  \def\y{0}

  \pill{\y}{softcream}{I}{Fundamentos}
  \pgfmathsetmacro\y{\y+\step}
  \pill{\y}{softpink}{II}{Arquitectura}
  \pgfmathsetmacro\y{\y+\step}
  \pill{\y}{softblue}{III}{Fases de Implementación}
  \pgfmathsetmacro\y{\y+\step}
  \pill{\y}{softpeach}{IV}{Paradigmas}
  \pgfmathsetmacro\y{\y+\step}
  \pill{\y}{softyellow}{V}{Evaluación y metricas}
  \pgfmathsetmacro\y{\y+\step}
  \pill{\y}{softgreen}{VI}{Futuro de RAG}
  
\end{tikzpicture}
\end{center}
\caption{Resumen esquemático de RAG}
\label{fig:secciones-rag}
\end{figure}

A partir de esta hoja de ruta se desarrolla un esquema más detallado, en el que se expone primero exploraremos su teoría, 
características y aplicaciones, como se muestra en la Fig \ref{fig:Fudamentos}. Luego profundizamos en su arquitectura en la cual 
se describe cada uno de los componentes que lo conforman (retriever, augmented y generation) y las variantes y mejoras que existen de cada uno. 
Posteriormente, se detalla el proceso de implementación, desde la preparación de datos hasta el componente de generación, incluyendo las técnicas y herramientas más relevantes.
A continuación en la Fig , se examinan los paradigmas de RAG, dando a conocer los tipos de paradigmas y sus clases, para luego en la Fig , se presentan las métricas y evaluadores automáticos 
utilizados en la evaluación de sistemas RAG, así como las consideraciones éticas y de equidad que deben tenerse en cuenta.
Finalmente, se discuten las tendencias emergentes, los desafíos que actualmente se tienen y futuras direcciones que podrían tomar los sistemas RAG.



% --- Grafico fundamentos
\begin{figure}[H]
\begin{center}
\begin{tikzpicture}[x=1cm,y=1cm]
\tikzset{
  mindoval/.style={
    rounded corners=18pt,
    draw=black!70,
    line width=0.7pt,
    minimum width=3.2cm,
    minimum height=1.1cm,
    inner sep=6pt,
    fill=white      % <<< todos los nodos en blanco
  },
  mindcenter/.style={
    rounded corners=8pt,
    draw=black!70,
    line width=0.9pt,
    minimum width=3.6cm,
    minimum height=1.1cm,
    inner sep=4pt,
    fill=softrose,  % <<< centro en morado
    font=\bfseries
  },
  mindapp/.style={
    mindoval,
    fill=softblue   % <<< nodo aplicación en azul
  },
  mindarrow/.style={-latex, line width=0.9pt}
}

% --- Nodo central
\node[mindcenter] (fund) at (0,0) {Fundamentos};

% --- Nodos arriba
\node[mindoval] (aport)  at (0,3) {Aportes de RAGs};
\node[mindoval] (ragft)  at (5,0) {RAG vs fine tuning};

% --- Nodos izquierda
\node[mindoval] (rolrag) at (-5,2.5) {Rol de RAG};
\node[mindoval] (qrag)   at (-5,0.75) {¿Qué es RAG?};
\node[mindoval] (qllm)   at (-5,-1.0) {¿Qué es un LLM?};

% --- Nodos abajo
\node[mindapp, minimum width=3.0cm] (apli) at (0,-3.0) {Aplicación};
\node[mindoval] (texto)  at (-5,-4.5) {Texto};
\node[mindoval] (img)    at (-2,-6) {Imagen};
\node[mindoval, minimum width=3.6cm] (audio)  at (2,-6) {Audio y video};
\node[mindoval] (codigo) at (5,-4.5) {Código};

% --- Flechas (tocando bordes de nodos)
\draw[mindarrow] (fund.north) -- (aport.south);
\draw[mindarrow] (fund.east) -- (ragft.west);
\draw[mindarrow] (fund.west) -- (rolrag.east);
\draw[mindarrow] (fund.west) -- (qrag.east);
\draw[mindarrow] (fund.west) -- (qllm.east);
\draw[mindarrow] (fund.south) -- (apli.north);

\draw[mindarrow] (apli.west) -- (texto.east);
\draw[mindarrow] (apli) -- (img.north);
\draw[mindarrow] (apli) -- (audio.north);
\draw[mindarrow] (apli.east) -- (codigo.west);

\end{tikzpicture}
\end{center}
\caption{Fundamentos de RAG}
\label{fig:Fudamentos}
\end{figure}

\subsubsection{Fundamentos}
en esta subseccion se va tratar 
\paragraph{Que es un llm}
explico
\paragraph{QUes un rag}
explico 
y asi describo