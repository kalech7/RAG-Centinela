\chapter{Pruebas, Resultados, Conclusiones y Recomendaciones}\label{ch:pruebas}
En este capítulo se presenta la fase de validación experimental expuesta en
la Sección~\ref{subsec:evaluacion}.
 Se detallan las pruebas realizadas, se analizan los resultados obtenidos y, finalmente, se exponen las conclusiones y
recomendaciones derivadas del estudio.
\section{Pruebas}\label{sec:demostracion}
La Tabla~\ref{tab:summary_pruebas} presenta una visión general del diseño experimental empleado para evaluar el sistema,
tal como se describen en la Sección~\ref{subsec:evaluacion}.
En este resumen se sintetizan las características del conjunto de pruebas, las configuraciones comparadas y las métricas utilizadas,
cuyos procedimientos completos se encuentran descritos a detalle en la sección correspondiente.
\begin{table}[H]
    \centering
    \renewcommand{\arraystretch}{1.15}
    \small
    \begin{tabular}{@{}l p{10.5cm}@{}}
        \toprule
        \textbf{Bloque} & \textbf{Resumen} \\
        \midrule

        \textbf{Conjunto de pruebas} &
        15 consultas totales: 5 \textit{single-hop}, 5 \textit{multi-hop}, 5 \textit{no answer}.
        Incluyen ground truth manual con evidencia relevante y respuesta ideal. \\

        \textbf{Condiciones evaluadas} &
        \begin{itemize}[leftmargin=1em,itemsep=0.1em]
            \item \textbf{LLM-solo}: sin recuperación; línea base.
            \item \textbf{RAG-básico}: Recuperador + concatenación top-$k$.
            \item \textbf{RAG-completo}: Recuperador + reranking + context builder.
        \end{itemize} \\

        \textbf{Retrieval} &
        Recall@15 y MRR para medir: (i) cobertura de evidencia en el top-$k$,
        (ii) posición del primer documento relevante. \\

        \textbf{Augmentation} &
        Ganancia: $\Delta\mathcal{M} = \mathcal{M}(\text{RAG}) - \mathcal{M}(\text{LLM-solo})$,
        que refleja el aporte del contexto recuperado. \\

        \textbf{Generation (LLM-as-Judge)} &
        Evaluación en dos etapas:
        \begin{itemize}[leftmargin=1em,itemsep=0.1em]
            \item Fidelidad al contexto: \textit{faithfulness}, cobertura de citas.
            \item Acuerdo con la respuesta ideal: exactitud, cobertura semántica, coherencia (0–5), con 3 repeticiones por consulta.
        \end{itemize} \\

        \textbf{Agregación y consistencia} &
        Índice global por consulta $A_i$ y macro-promedio $\overline{A}$.
        Consistencia mediante Cohen's $\kappa$ (acuerdo binarizado con umbral $A_i \ge 4$). \\

        \textbf{Operación} &
        Latencia end-to-end por consulta (recuperación + reranking + contexto + generación). \\

        \bottomrule
    \end{tabular}
    \caption{Resumen general del diseño experimenta.}
    \label{tab:summary_pruebas}
\end{table}


\section{Resultados}\label{sec:evaluacion-del-desempeno}

estos son los resultados con los valores que obtuve y se puede presentar en un diagrama
\section{Discusión}\label{sec:resultados}

explico los resultados los describo
\section{Conclusiones}\label{sec:conclusiones}


\section{Recomendaciones}\label{sec:recomendaciones}
