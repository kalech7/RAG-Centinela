\chapter{Metodología}
\section{Revisión sistemática}
La presente revisión se fundamenta en una metodología combinada que integra la estructura de 
PRISMA-P (Preferred Reporting Items for Systematic Review and Meta-Analysis Protocols) con el enfoque de 
Umbrella Review, según los lineamientos del Instituto Joanna Briggs (JBI). Esta combinación metodológica 
permite una revisión de revisiones sistemáticas de alto nivel, asegurando tanto la planificación rigurosa 
como la ejecución estructurada del proceso.

El Umbrella Review, de acuerdo con JBI, es un tipo de revisión sistemática que sintetiza 
evidencia secundaria, es decir, revisiones sistemáticas y metaanálisis ya publicados, y no 
estudios primarios. Su objetivo es consolidar conocimientos, identificar áreas con consenso,
 detectar contradicciones entre estudios secundarios y revelar vacíos en la evidencia. Este 
 enfoque requiere, como toda revisión sistemática, la elaboración de un protocolo previo, 
 el cual debe detallar la pregunta de investigación, los criterios de inclusión/exclusión, 
 las estrategias de búsqueda, el método de síntesis y otros aspectos clave. Sin embargo, 
 JBI no prescribe un formato único para este protocolo.

Aquí es donde PRISMA-P aporta un marco estandarizado y ampliamente validado para el 
desarrollo del protocolo. PRISMA-P está diseñado específicamente para redactar protocolos 
de revisiones sistemáticas, incluyendo revisiones tipo umbrella. Este proporciona una lista
de verificación de 17 ítems que deben cubrirse antes de iniciar la revisión: justificación,
objetivos, criterios, fuentes, métodos de selección, evaluación de calidad, estrategia de 
extracción y síntesis, entre otros. \\Al seguir PRISMA-P, se garantiza que la planificación 
del estudio umbrella sea transparente, completa, reproducible y libre de sesgos metodológicos. 
PRISMA-P actúa principalmente en la Fase 1: Planificación estructurada, donde permite definir de forma clara y anticipada todos los componentes del protocolo necesarios para ejecutar la revisión. Durante esta fase, se establecen la justificación del estudio, la pregunta de investigación, los objetivos, los criterios de inclusión/exclusión, las fuentes de información, la estrategia de búsqueda, el procedimiento para seleccionar los estudios, el método de evaluación de calidad (en este caso, usando herramientas del JBI),
el plan de extracción de datos y el método de síntesis. Esta estructura da lugar a un protocolo sólido, 
que puede ser registrado públicamente (por ejemplo, en PROSPERO) si se desea, y asegura que el proceso esté claramente documentado antes de su ejecución.

Una vez planificado el protocolo con PRISMA-P, se procede a la Fase 2: 
Ejecución, basada en las directrices metodológicas de Umbrella Review del JBI. En esta etapa se ejecuta la búsqueda definida, se seleccionan las revisiones sistemáticas relevantes, se evalúa su calidad metodológica con las herramientas del JBI, se extrae la información clave, se sintetizan los hallazgos de manera estructurada y se presentan los resultados con claridad visual y narrativa. El hecho de haber definido previamente todos estos pasos con PRISMA-P garantiza que la revisión se lleve a cabo con consistencia y sin desviaciones metodológicas, como exige JBI.

\subsection*{Fase 1: Planificación estructurada (PRISMA-P)}

\begin{enumerate}
    \item \textbf{Justificación y formulación de la pregunta}

    Se define la relevancia de realizar una revisión de revisiones (umbrella). La pregunta de investigación se formula utilizando marcos como \textbf{PICO} (Población, Intervención, Comparador, Resultado) o \textbf{PICo} (Población, Fenómeno de interés, Contexto), según el enfoque del estudio.

    \item \textbf{Objetivos específicos}

    Se establecen los objetivos generales y específicos que guían el proceso de búsqueda, selección y síntesis de la evidencia.

    \item \textbf{Criterios de inclusión y exclusión}

    Se determinan anticipadamente criterios claros y coherentes. Por ejemplo, se incluyen solo revisiones sistemáticas publicadas entre 2015 y 2024, en inglés o español, con evaluación metodológica explícita.

    \item \textbf{Fuentes de información y estrategia de búsqueda}

    Se seleccionan las bases de datos (por ejemplo: Scopus, Web of Science, IEEE Xplore, Google Scholar) y se construyen estrategias de búsqueda reproducibles utilizando operadores booleanos y filtros específicos.

    \item \textbf{Método de selección de estudios}

    Se establece un procedimiento doble de revisión (dos evaluadores independientes) para la selección de estudios, con un mecanismo claro de resolución de discrepancias.

    \item \textbf{Método para evaluar la calidad metodológica}

    Se especifica que se utilizarán las \textbf{herramientas del JBI} para evaluar la calidad metodológica de las revisiones sistemáticas seleccionadas.

    \item \textbf{Plan para la extracción de datos}

    Se define qué información se extraerá: autor, año, objetivo, tipo de revisión, número de estudios primarios incluidos, hallazgos clave, entre otros, utilizando formularios estructurados.

    \item \textbf{Plan de síntesis}

    Se establece que la síntesis será narrativa, tabular o visual. No se realizará metaanálisis nuevo, en coherencia con el enfoque umbrella, que no analiza datos primarios.
\end{enumerate}

\subsection*{Fase 2: Ejecución sistemática (Umbrella Review – JBI)}

\begin{enumerate}[resume]
    \item \textbf{Búsqueda sistemática ejecutada}

    Se realiza la búsqueda según lo especificado en el protocolo PRISMA-P. Se documentan bases utilizadas, fechas, términos empleados y número de resultados obtenidos.

    \item \textbf{Selección de revisiones sistemáticas y metaanálisis}

    Se aplican los criterios de inclusión/exclusión para seleccionar revisiones relevantes. Se recomienda presentar el proceso mediante un diagrama de flujo (tipo PRISMA).

    \item \textbf{Evaluación crítica de la calidad}

    Se aplica el \textbf{checklist del JBI} para valorar la calidad metodológica de cada revisión. Aquellas con baja calidad pueden ser excluidas o diferenciadas.

    \item \textbf{Extracción de información clave}

    De cada revisión se extraen los siguientes datos:

    \begin{itemize}
        \item Objetivo del estudio
        \item Tipo de intervención o contexto
        \item Número de estudios primarios incluidos
        \item Principales hallazgos
        \item Conclusiones generales
        \item Limitaciones reportadas
    \end{itemize}

    \item \textbf{Síntesis de resultados}

    Se organizan los hallazgos en:

    \begin{itemize}
        \item Tablas comparativas
        \item Resúmenes narrativos
        \item Análisis temáticos
    \end{itemize}

    Se identifican:
    \begin{itemize}
        \item Coincidencias entre revisiones
        \item Contradicciones o divergencias
        \item Vacíos de conocimiento
    \end{itemize}

    \item \textbf{Presentación visual y tabular}

    Se representan los resultados mediante:
    \begin{itemize}
        \item Matrices de evidencia
        \item Mapas temáticos
        \item Tablas resumen estructuradas
    \end{itemize}

    \item \textbf{Discusión y conclusiones}

    Se interpretan los hallazgos desde una perspectiva estratégica. Se discuten las fortalezas y limitaciones de la evidencia y se plantean implicaciones para la investigación futura o la práctica profesional.
\end{enumerate}










\section{Enfoque Design Science Research (DSR)}
De acuerdo con vom Brocke et al. \cite{vomBrocke2020}, Design Science Research, desarrollada en 1969, es un paradigma de resolución de problemas que busca mejorar el conocimiento humano mediante la creación de artefactos innovadores. En otras palabras, es una metodología que crea soluciones a problemas reales y, al mismo tiempo, genera conocimiento útil y aplicable sobre cómo diseñar estas soluciones. Las etapas que se aplicarán en el presente trabajo son las siguientes:

\begin{itemize}[align=left, label=--]
\item \textbf{Identificación del problema y motivación} \
En esta etapa se precisa el problema y se justifica por qué es necesaria una solución. De acuerdo con Peffers et al. (2008), esta etapa exige analizar el problema en detalle, descomponiéndolo en sus partes clave para identificar sus causas, efectos y alcance. Además, es crucial justificar la relevancia del problema, tanto desde una perspectiva teórica (es decir, cómo contribuye al conocimiento académico) como desde una perspectiva práctica (cómo afecta a organizaciones, usuarios o sistemas reales).
También implica explorar la literatura para verificar que el problema es relevante, desafiante y nuevo, lo que permite definir los límites del proyecto de investigación.

\item \textbf{Definir los objetivos para la solución} \
Se plantean los criterios que debe cumplir una solución exitosa basándose en el conocimiento existente y en la factibilidad técnica y organizacional.
Los objetivos deberán permitir construir algo efectivo y deseable, no solamente desde el ámbito académico sino también en el entorno en que se aplicará. Estos pueden expresarse en términos cualitativos o cuantitativos; el investigador establece aquí la meta hacia donde se dirigirá el artefacto.

\item \textbf{Diseño y desarrollo del artefacto} \
En esta etapa se construye una solución concreta, como un modelo, software o sistema, que responde directamente a los objetivos planteados. Para ello, se utiliza el conocimiento existente que fundamenta las decisiones del diseño y la estructura del artefacto. No solo se trata de crear algo, sino de asegurar que pueda ser comprendido, evaluado y replicado por otros.

\item \textbf{Demostración del uso del artefacto para resolver el problema} \
Se muestra cómo se usa el artefacto en un escenario real o simulado. Esta demostración no valida científicamente su efectividad, sino que muestra su aplicabilidad, evidenciando que el artefacto propuesto puede operar de forma efectiva.
Por su parte, vom Brocke et al. (2020) destacan que esta etapa es fundamental para conectar el diseño teórico con la realidad del usuario o del entorno organizacional, permitiendo detectar oportunidades de mejora antes de una evaluación rigurosa.

\item \textbf{Evaluación del desempeño del artefacto} \
Se mide su efectividad, eficiencia o impacto. El objetivo es obtener evidencia empírica o lógica que permita justificar el valor y la utilidad del artefacto.
En complemento, vom Brocke et al. (2020) plantean una visión más dinámica al introducir el concepto de evaluación formativa, que puede desarrollarse de forma continua a lo largo del proceso de investigación, no solo al final. Esta puede realizarse antes de su implementación en un entorno real o después de la misma, permitiendo ciclos iterativos de rediseño y mejora.

\item \textbf{Comunicación de los resultados al público académico y profesional} \
Finalmente, esta etapa consiste en difundir de forma clara los resultados del diseño y de la investigación realizada.
\end{itemize}

Estos pasos están basados en el modelo clásico de DSR de Peffers (2008), que vom Brocke adapta y expande en su guía.\\
\begin{landscape}
\begin{figure}[ht]
    \centering
    \resizebox{\linewidth}{!}{%
    \begin{tikzpicture}[scale=0.9, transform shape, node distance=1cm and 1.5cm,
        squarednode/.style={rectangle, draw=black, fill=white, very thick, minimum size=5mm, text width=4cm},
        everynode/.style={align=center}, 
        rectalbecirc/.style={
        draw=black,           % borde negro
        fill=white,           % fondo blanco
        line width=0.1pt,           % grosor del borde
        rounded corners=9pt,  % esquinas redondeadas
        align=center,         % alinear texto al centro
        text width=3cm,       % ancho del texto
        minimum height=1cm,   % altura mínima del nodo
        inner sep=6.5pt,
        font=\small
        }
        ]
        % nodos
        %horizontal
        \node[squarednode] (problema) {Identificación del problema};
        \node[squarednode] (objetivos) [right=of problema] {Objetivos para la solución};
        \node[squarednode] (diseno) [right=of objetivos] {Diseño y desarrollo del artefacto};
        \node[squarednode] (demostracion) [right=of diseno] {Demostración};
        \node[squarednode] (evaluacion) [right=of demostracion] {Evaluación del desempeño};
        \node[squarednode] (comunicacion) [right=of evaluacion] {Comunicación de resultados};
        %vertical
        \node[rectalbecirc] (start) [below=of problema] {Centinela tiene que presentar mejores resultados en las respuestas al momento de buscar información y generar contenido de valor para el usuario.};
        \node[rectalbecirc] (obj) [below=of objetivos] {
            \vspace*{-\baselineskip} % reduce el espacio inicial de la lista
            \begin{itemize} [left=0pt]\setlength\itemsep{1pt}\setlength\leftmargini{1pt}
                \renewcommand\labelitemi{\tiny$\bullet$}
                \item Mejorar la precisión de las respuestas.
                \item Aumentar la relevancia del contenido generado.
                \item Optimizar la experiencia del usuario.
            \end{itemize}
        };
        \node[rectalbecirc] (design) [below=of diseno] {Se contempla un diseño que integra retriver, augmented y generation lo que da como resultado un buscador el cual su respuesta es en lenguaje natural.};
        \node[rectalbecirc] (dem) [below=of demostracion] {Se demuestra el uso del buscador en un entorno controlado, mostrando su capacidad para responder preguntas y generar contenido relevante.};
        \node[rectalbecirc] (eval) [below=of evaluacion] {Se evalúa el buscador mediante métricas estándar y FATE};
        \node[rectalbecirc] (com) [below=of comunicacion] {En el presente trabajo se presenta el proceso que se ha realizado para el desarrollo del buscador, así como los resultados obtenidos y recomendaciones para futuras mejoras.};

        %conexiones horizontales
        \draw[-to] (problema) -- (objetivos);
        \draw[-to] (objetivos) -- (diseno);
        \draw[-to] (diseno) -- (demostracion);
        \draw[-to] (demostracion) -- (evaluacion);
        \draw[-to] (evaluacion) -- (comunicacion);
        %conexiones verticales
        \draw[-to] (problema) -- (start);
        \draw[-to] (objetivos) -- (obj);
        \draw[-to] (diseno) -- (design);    
        \draw[-to] (demostracion) -- (dem);
        \draw[-to] (evaluacion) -- (eval);
        \draw[-to] (comunicacion) -- (com);

        %contenedor
        \node[draw=black, rounded corners=10pt, thick, minimum width=23cm, minimum height=4cm, fill=gray!5, anchor=north, name=contenedor] 
at ([yshift=-0.75cm]current bounding box.south) {};
        \coordinate (center) at (contenedor.center);
        \def\sep{7} % Separación horizontal
        % Elipses dentro del contenedor
        \node[draw=black, ellipse, minimum width=3cm, minimum height=2.2cm, fill=blue!10, align=center, font=\small] 
            (elipse1) at ($(center) + (-\sep,0)$) 
            {Definición de RAG};

        \node[draw=black, ellipse, minimum width=3cm, minimum height=2.2cm, fill=green!10, align=center, font=\small] 
            (elipse2) at ($(center)$) 
            {Arquitectura de RAG};

        \node[draw=black, ellipse, minimum width=3cm, minimum height=2.2cm, fill=orange!10, align=center, font=\small] 
            (elipse3) at ($(center) + (\sep,0)$) 
            {Optimización de RAG};
        \node[align=center, text width=30cm, yshift=-1.5cm] at (center) {\fontsize{10}{12}\selectfont
    Puntos de entrada a la investigación};

        % Conexiones entre elipses y nodos
        \draw[-to] (elipse1) -- (start);
        \draw[-to] (elipse1) -- (obj);
        \draw[-to] (elipse2) -- (design);
        \draw[-to] (elipse3) -- (eval);
        \draw[-to] (elipse3) -- (dem);
        \draw[-to, bend right=20] (comunicacion) to node[pos=0.06, above,sloped]{Iterativo} (objetivos);
        \draw[-to, bend right=20] (comunicacion) to (diseno);
        \draw[-to, bend right=20] (comunicacion) to (evaluacion);
    \end{tikzpicture}
    }
    \caption{Proceso de Diseño de Investigación para el desarrollo de RAG en Centinela}
\end{figure}
\end{landscape}

