\chapter{Metodologia}
\section{Revision sistematica}
Una revision sistematica de la literatura es un estudio 
que utiliza una metodologia para seleccionar, analizar y 
sintentizar toda la invetigacion relevante que responda una pregunta de investigacion 
Este tipo de metodología garantiza la validez científica y favorece su reproducibilidad, lo que permite que otros investigadores verifiquen los resultados 
y repliquen el proceso. En este texto se adopta un enfoque combinado que integra una revision sistematica
tipo paraguas y la declaracion PRISMA 2020.\\
Umbrella SLR permite ordenar y jerarquizar la información científica disponible al recopilar y evaluar sistemáticamente evidencia de múltiples revisiones sistemáticas y 
meta-análisis sobre un mismo tema \cite{Papatheodorou2019}. Este tipo de revision ofrece una forma de identificar patrones consistentes,contradicciones y vacios en la literatura. 
Ademas, su utlizacion se combinan perfectamente con la guia PRISMA ya que ambas garantizan que el proceso sea transparente y reproducible.\\
En este texto, se siguio el PRISMA 2020 Checklist, el cual define los ítems esenciales para reportar una revisión sistemática de manera clara y estructurada.
La estructura PRISMA se utliza para efinir las etapas de búsqueda, selección, evaluación y reporte, mientras que el enfoque paraguas permitió organizar y sintentizar el conocimiento acumulado de forma jerarquica.
\section{Enfoque Design Science Research (DSR)}
Design Science Research desarollada en 1969, es un paradigma de resolución de problemas que busca mejorar el 
conocimiento humano mediante la creación de artefactos innovadores. \cite{vomBrocke2020} En otras palabras, es una metodologa que crea soluciones a problemas  reales y al mismo tiempo generar
conocimiento util y aplicable sobre como diseñar estas soluciones. Las etapas son las siguientes que se aplicaran: 
\begin{itemize}[align=left, label=--]
    \item \textbf{Identificación del problema y motivación} \\
    En esta actividad se precisa el problema y se justifica por qué es necesaria una solucion. De acuerdo con Peffers et al. (2008), esta etapa exige analizar el problema en detalle, descomponiéndolo en sus partes clave para identificar sus causas, efectos y alcance. Además, es crucial justificar la relevancia del problema, tanto desde una perspectiva teórica (es decir, cómo contribuye al conocimiento académico) como desde una perspectiva práctica (cómo afecta a organizaciones, usuarios o sistemas reales). 
    Tambien en esta etapa implica explorar la literatura para verificar que el problema es relvante, desafiante y nuevo lo que 
    permite definir los limites del proyecto de investigacion.
    \item \textbf{Definir los objetivos para la solucion}\\
    Se plantean los criterios que debe cumplir una solución exitosa basandose en el conocimineto existente y en la factibilidad tecnica y organizacional.
    Los objetivos deberan permitir construir algo efectivo y deseable no solamente desde el ambito academico sino tambien en el entorno que se aplicara; Estos pueden expresarse en terminos cualitativos o cuantitativos, el investigador establece aqui la meta hacia donde se dirigira el artefacto.
    \item \textbf{Diseño y desarrollo del artefacto}\\
    Se construye una solución concreta, como un modelo, software o sistema.
    
    \item \textbf{Demostración del uso del artefacto para resolver el problema}\\
    Se muestra cómo se usa el artefacto en un escenario real o simulado.
    \item \textbf{Evaluación del desempeño del artefacto}\\
    Se mide su efectividad, eficiencia o impacto.
    \item \textbf{Comunicación de los resultados al público académico y profesional}\\
    Se documentan y difunden los resultados del estudio y del diseño.
\end{itemize}
Estos pasos están basados en el modelo clásico de DSR de Peffers (2008), que vom Brocke adaptan y expanden en su guía.

