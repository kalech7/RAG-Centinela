\chapter{Metodologia}
\section{Revision sistematica}
Una revision sistematica de la literatura es un estudio 
que utiliza una metodologia para seleccionar, analizar y 
sintentizar toda la invetigacion relevante que responda una pregunta de investigacion 
Este tipo de metodología garantiza la validez científica y favorece su reproducibilidad, lo que permite que otros investigadores verifiquen los resultados 
y repliquen el proceso. En este texto se adopta un enfoque combinado que integra una revision sistematica
tipo paraguas y la declaracion PRISMA 2020.\\
Umbrella SLR permite ordenar y jerarquizar la información científica disponible al recopilar y evaluar sistemáticamente evidencia de múltiples revisiones sistemáticas y 
meta-análisis sobre un mismo tema \cite{Papatheodorou2019}. Este tipo de revision ofrece una forma de identificar patrones consistentes,contradicciones y vacios en la literatura. 
Ademas, su utlizacion se combinan perfectamente con la guia PRISMA ya que ambas garantizan que el proceso sea transparente y reproducible.\\
En este texto, se siguio el PRISMA 2020 Checklist, el cual define los ítems esenciales para reportar una revisión sistemática de manera clara y estructurada.
La estructura PRISMA se utliza para efinir las etapas de búsqueda, selección, evaluación y reporte, mientras que el enfoque paraguas permitió organizar y sintentizar el conocimiento acumulado de forma jerarquica.
\section{Enfoque Design Science Research (DSR)}
Design Science Research desarollada en 1969, es un paradigma de resolución de problemas que busca mejorar el 
conocimiento humano mediante la creación de artefactos innovadores. \cite{vomBrocke2020} En otras palabras, es una metodologa que crea soluciones a problemas  reales y al mismo tiempo generar
conocimiento util y aplicable sobre como diseñar estas soluciones. Las etapas son las siguientes que se aplicaran: 
\begin{itemize}
    \item Identificación del problema y motivación
    \item Definir los objetivos para la solucion
    \item Diseño y desarrollo del artefacto
    \item Demostración del uso del artefacto para resolver el problema
    \item Evaluación del desempeño del artefacto
    \item Comunicación de los resultados al público académico y profesional
\end{itemize}
Estos pasos están basados en el modelo clásico de DSR de Peffers (2007), que vom Brocke adaptan y expanden en su guía.

