\chapter{Metodología}
\section{Revisión sistemática}
Una revisión sistemática de la literatura es un estudio que utiliza una metodología para seleccionar, analizar y sintetizar toda la investigación relevante que responda a una pregunta de investigación. Este tipo de metodología garantiza la validez científica y favorece su reproducibilidad, lo que permite que otros investigadores verifiquen los resultados y repliquen el proceso.
En este texto se adopta un enfoque combinado que integra una revisión sistemática tipo umbrella \cite{peffers2008design} y la declaración PRISMA 2020 (cita de PRISMA).
Umbrella SLR permite ordenar y jerarquizar la información científica disponible al recopilar y evaluar sistemáticamente evidencia de múltiples revisiones sistemáticas y meta-análisis sobre un mismo tema \cite{Papatheodorou2019}. Este tipo de revisión ofrece una forma de identificar patrones consistentes, contradicciones y vacíos en la literatura. Además, su utilización se combina perfectamente con la guía PRISMA, ya que ambas garantizan que el proceso sea transparente y reproducible.

Para el presente trabajo en TIC, se siguió la lista de verificación PRISMA 2020, la cual define los ítems esenciales para reportar una revisión sistemática de manera clara y estructurada. La estructura PRISMA se utiliza para definir las etapas de búsqueda, selección, evaluación y reporte, mientras que el enfoque Umbrella SLR permitió organizar y sintetizar el conocimiento acumulado de forma jerárquica y
además, el Instituto Joanna Briggs (JBI) establece un protocolo riguroso que incluye pasos como la formulación de la pregunta de investigación mediante la estrategia PICO o PICo, la evaluación crítica de la calidad metodológica de las revisiones incluidas, y la presentación tabular de los resultados cualitativos o cuantitativos con resúmenes visuales tipo “semáforo”.

\begin{itemize}[align=left, label=--]
\item \textbf{Título} 
\item \textbf{}
\end{itemize}













\section{Enfoque Design Science Research (DSR)}
De acuerdo con vom Brocke et al. \cite{vomBrocke2020}, Design Science Research, desarrollada en 1969, es un paradigma de resolución de problemas que busca mejorar el conocimiento humano mediante la creación de artefactos innovadores. En otras palabras, es una metodología que crea soluciones a problemas reales y, al mismo tiempo, genera conocimiento útil y aplicable sobre cómo diseñar estas soluciones. Las etapas que se aplicarán en el presente trabajo son las siguientes:

\begin{itemize}[align=left, label=--]
\item \textbf{Identificación del problema y motivación} \
En esta etapa se precisa el problema y se justifica por qué es necesaria una solución. De acuerdo con Peffers et al. (2008), esta etapa exige analizar el problema en detalle, descomponiéndolo en sus partes clave para identificar sus causas, efectos y alcance. Además, es crucial justificar la relevancia del problema, tanto desde una perspectiva teórica (es decir, cómo contribuye al conocimiento académico) como desde una perspectiva práctica (cómo afecta a organizaciones, usuarios o sistemas reales).
También implica explorar la literatura para verificar que el problema es relevante, desafiante y nuevo, lo que permite definir los límites del proyecto de investigación.

\item \textbf{Definir los objetivos para la solución} \
Se plantean los criterios que debe cumplir una solución exitosa basándose en el conocimiento existente y en la factibilidad técnica y organizacional.
Los objetivos deberán permitir construir algo efectivo y deseable, no solamente desde el ámbito académico sino también en el entorno en que se aplicará. Estos pueden expresarse en términos cualitativos o cuantitativos; el investigador establece aquí la meta hacia donde se dirigirá el artefacto.

\item \textbf{Diseño y desarrollo del artefacto} \
En esta etapa se construye una solución concreta, como un modelo, software o sistema, que responde directamente a los objetivos planteados. Para ello, se utiliza el conocimiento existente que fundamenta las decisiones del diseño y la estructura del artefacto. No solo se trata de crear algo, sino de asegurar que pueda ser comprendido, evaluado y replicado por otros.

\item \textbf{Demostración del uso del artefacto para resolver el problema} \
Se muestra cómo se usa el artefacto en un escenario real o simulado. Esta demostración no valida científicamente su efectividad, sino que muestra su aplicabilidad, evidenciando que el artefacto propuesto puede operar de forma efectiva.
Por su parte, vom Brocke et al. (2020) destacan que esta etapa es fundamental para conectar el diseño teórico con la realidad del usuario o del entorno organizacional, permitiendo detectar oportunidades de mejora antes de una evaluación rigurosa.

\item \textbf{Evaluación del desempeño del artefacto} \
Se mide su efectividad, eficiencia o impacto. El objetivo es obtener evidencia empírica o lógica que permita justificar el valor y la utilidad del artefacto.
En complemento, vom Brocke et al. (2020) plantean una visión más dinámica al introducir el concepto de evaluación formativa, que puede desarrollarse de forma continua a lo largo del proceso de investigación, no solo al final. Esta puede realizarse antes de su implementación en un entorno real o después de la misma, permitiendo ciclos iterativos de rediseño y mejora.

\item \textbf{Comunicación de los resultados al público académico y profesional} \
Finalmente, esta etapa consiste en difundir de forma clara los resultados del diseño y de la investigación realizada.
\end{itemize}

Estos pasos están basados en el modelo clásico de DSR de Peffers (2008), que vom Brocke adapta y expande en su guía.\\
\begin{landscape}
\begin{figure}[ht]
    \centering
    \resizebox{\linewidth}{!}{%
    \begin{tikzpicture}[scale=0.9, transform shape, node distance=1cm and 1.5cm,
        squarednode/.style={rectangle, draw=black, fill=white, very thick, minimum size=5mm, text width=4cm},
        everynode/.style={align=center}, 
        rectalbecirc/.style={
        draw=black,           % borde negro
        fill=white,           % fondo blanco
        line width=0.1pt,           % grosor del borde
        rounded corners=9pt,  % esquinas redondeadas
        align=center,         % alinear texto al centro
        text width=3cm,       % ancho del texto
        minimum height=1cm,   % altura mínima del nodo
        inner sep=6.5pt,
        font=\small
        }
        ]
        % nodos
        %horizontal
        \node[squarednode] (problema) {Identificación del problema};
        \node[squarednode] (objetivos) [right=of problema] {Objetivos para la solución};
        \node[squarednode] (diseno) [right=of objetivos] {Diseño y desarrollo del artefacto};
        \node[squarednode] (demostracion) [right=of diseno] {Demostración};
        \node[squarednode] (evaluacion) [right=of demostracion] {Evaluación del desempeño};
        \node[squarednode] (comunicacion) [right=of evaluacion] {Comunicación de resultados};
        %vertical
        \node[rectalbecirc] (start) [below=of problema] {Centinela tiene que presentar mejores resultados en las respuestas al momento de buscar información y generar contenido de valor para el usuario.};
        \node[rectalbecirc] (obj) [below=of objetivos] {
            \vspace*{-\baselineskip} % reduce el espacio inicial de la lista
            \begin{itemize} [left=0pt]\setlength\itemsep{1pt}\setlength\leftmargini{1pt}
                \renewcommand\labelitemi{\tiny$\bullet$}
                \item Mejorar la precisión de las respuestas.
                \item Aumentar la relevancia del contenido generado.
                \item Optimizar la experiencia del usuario.
            \end{itemize}
        };
        \node[rectalbecirc] (design) [below=of diseno] {Se contempla un diseño que integra retriver, augmented y generation lo que da como resultado un buscador el cual su respuesta es en lenguaje natural.};
        \node[rectalbecirc] (dem) [below=of demostracion] {Se demuestra el uso del buscador en un entorno controlado, mostrando su capacidad para responder preguntas y generar contenido relevante.};
        \node[rectalbecirc] (eval) [below=of evaluacion] {Se evalúa el buscador mediante métricas estándar y FATE};
        \node[rectalbecirc] (com) [below=of comunicacion] {En el presente trabajo se presenta el proceso que se ha realizado para el desarrollo del buscador, así como los resultados obtenidos y recomendaciones para futuras mejoras.};

        %conexiones horizontales
        \draw[-to] (problema) -- (objetivos);
        \draw[-to] (objetivos) -- (diseno);
        \draw[-to] (diseno) -- (demostracion);
        \draw[-to] (demostracion) -- (evaluacion);
        \draw[-to] (evaluacion) -- (comunicacion);
        %conexiones verticales
        \draw[-to] (problema) -- (start);
        \draw[-to] (objetivos) -- (obj);
        \draw[-to] (diseno) -- (design);    
        \draw[-to] (demostracion) -- (dem);
        \draw[-to] (evaluacion) -- (eval);
        \draw[-to] (comunicacion) -- (com);

        %contenedor
        \node[draw=black, rounded corners=10pt, thick, minimum width=23cm, minimum height=4cm, fill=gray!5, anchor=north, name=contenedor] 
at ([yshift=-0.75cm]current bounding box.south) {};
        \coordinate (center) at (contenedor.center);
        \def\sep{7} % Separación horizontal
        % Elipses dentro del contenedor
        \node[draw=black, ellipse, minimum width=3cm, minimum height=2.2cm, fill=blue!10, align=center, font=\small] 
            (elipse1) at ($(center) + (-\sep,0)$) 
            {Fase de exploración};

        \node[draw=black, ellipse, minimum width=3cm, minimum height=2.2cm, fill=green!10, align=center, font=\small] 
            (elipse2) at ($(center)$) 
            {Fase de desarrollo};

        \node[draw=black, ellipse, minimum width=3cm, minimum height=2.2cm, fill=orange!10, align=center, font=\small] 
            (elipse3) at ($(center) + (\sep,0)$) 
            {Fase de resultados};
        \node[align=center, text width=30cm, yshift=-1.5cm] at (center) {\fontsize{10}{12}\selectfont
    Texto explicativo sobre las fases...};


    \end{tikzpicture}
    }
    \caption{Proceso de Diseño de Investigación para el desarrollo de RAG en Centinela}
\end{figure}
\end{landscape}

