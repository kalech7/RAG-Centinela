\chapter{Metodología}
\section{Revisión sistemática}
La Umbrella Review, según los lineamientos del Instituto Joanna Briggs (JBI), es un tipo de revisión sistemática que recopila y analiza evidencia secundaria, 
es decir, revisiones sistemáticas y metaanálisis ya publicados.
Su propósito es consolidar el conocimiento disponible, identificar coincidencias y contradicciones en la literatura existente, así como señalar vacíos de evidencia.
Para ello, requiere la elaboración de un protocolo previo que establezca criterios de inclusión y exclusión, estrategias de búsqueda y métodos de síntesis, garantizando 
un proceso transparente y riguroso.

Por otra parte, la estrategia de propagación de citaciones (Back-and-Forward Citation Propagating) complementa este enfoque al permitir encontrar dinámicamente la literatura. 
A través de la propagación de citaciones se amplía y actualiza la literatura encontrada en las bases de datos tradicionales. De este modo, se superan limitaciones como la 
indexación incompleta, las variaciones en el uso de palabras clave o la exclusión de ciertas publicaciones.


\subsection*{Metodología: Umbrella Review con Propagación de Citaciones}
Como parte de la metodología Umbrella Review es necesario establecer un protocolo para ejecutar la revisión.
Se han considerado las siguientes fases para dicho protocolo:
\begin{enumerate}
    \item \textbf{Propósito de la revisión}  
    
    La revisión se justifica en la necesidad de consolidar evidencia secundaria de calidad, aprovechando el enfoque de propagación de citaciones 
    para garantizar una búsqueda amplia, estructurada y actualizada.

    \item \textbf{Objetivos específicos}  

    Se definen los objetivos generales y específicos que guiarán la identificación de literatura mediante la propagación de citaciones, 
    así como el proceso de síntesis de resultados.

    \item \textbf{Criterios de inclusión y exclusión}  
    Se definen de manera general como la incorporación de revisiones y metaanálisis que sean pertinentes, de calidad y relacionados con el tema de estudio, 
    y la exclusión de aquellos trabajos que no cumplan con estos requisitos de relevancia.
    
    \item \textbf{Identificación del estudio semilla y propagación de citaciones}  
    
    La búsqueda se inicia en bases de datos académicas como \textit{Scopus}, \textit{Web of Science}, \textit{IEEE Xplore} o \textit{Google Scholar}, a fin de localizar 
    un estudio semilla (revisión o resumen amplio) que ofrezca una cobertura representativa del tema.  
    A partir de este estudio, se aplica la estrategia de Back-and-Forward Citation Propagation, que combina:  
    \begin{itemize}
        \item \textit{Backward citation:} revisión de las referencias citadas en el estudio semilla.  
        \item \textit{Forward citation:} identificación de trabajos más recientes que citan al estudio semilla.  
    \end{itemize}
    De este modo, el corpus de literatura se amplía progresivamente hasta alcanzar un punto de saturación en el que la propagación deja de aportar nueva evidencia relevante.

    \item \textbf{Selección de revisiones relevantes}  

    A partir de la propagación de citaciones, se aplican los criterios de inclusión - exclusión para determinar qué revisiones serán incorporadas al análisis.
    

    \item \textbf{Valoración de la calidad de la evidencia}  

    La calidad de los estudios se evalúa según los criterios definidos, garantizando su consistencia al tema de estudio.
    Para apoyar este proceso se usa una herramienta de análisis que facilite la organización y valoración sistemática de la evidencia.

    \item \textbf{Extracción de información clave}  

    De cada revisión seleccionada se extraerán datos esenciales, organizados en una tabla de extracción que incluirá:  
    \begin{itemize}
        \item Autor y año de publicación  
        \item Objetivo del estudio  
        \item Tipo de revisión  
        \item Número de estudios primarios incluidos  
        \item Principales hallazgos  
        \item Conclusiones generales  
        \item Limitaciones reportadas  
    \end{itemize}

    \item \textbf{Síntesis y representación de resultados}  

    Los hallazgos se organizarán en dos niveles complementarios:  
    \begin{itemize}
        \item \textbf{Tabular:} tablas comparativas de las revisiones incluidas.
        \item \textbf{Narrativo:} síntesis descriptiva de los principales hallazgos.  
        \item \textbf{Temático y visual:} mapas de evidencia y esquemas que reflejen la propagación de citaciones, mostrando las conexiones entre estudios clave.  
    \end{itemize}

    \item \textbf{Discusión y conclusiones}  

    Los resultados se interpretan desde una perspectiva crítica, destacando fortalezas, limitaciones y la evolución de la evidencia en el tiempo. Se identifican coincidencias y divergencias entre revisiones, así como vacíos de conocimiento, y se proponen líneas de investigación futura.
\end{enumerate}
En esta metodología, el Umbrella Review se utiliza como marco general para sintetizar evidencia secundaria a partir de revisiones de exhaustivas de la literatura, 
complementándose 
con la propagación de citaciones para integrar aportes recientes y 
reflejar la evolución del conocimiento disponible.


\subsection{Ejecución del estudio de la literatura}
\subsubsection{Propósito y objetivos de la revisión}
El propósito de esta revisión es consolidar la información disponible sobre los RAG, abordando su estudio desde los fundamentos hasta las fases de desarrollo. 
Se inicia con su definición y arquitectura, para luego profundizar en las etapas clave del proceso: Extraccion del corpus, preprocesamiento, vectorización, recuperación de información, 
evaluación, almacenamiento en bases vectoriales y generación de resultados. Asimismo, se examinan los paradigmas, las métricas de evaluación y el futuro de RAG.
Durante esta revision se busca lograr el objetivo general de proporcionar un panorama global y actualizado sobre los RAG,
exponiendo sus fundamentos, desarollo y aplicación.

\subsubsection{Criterios de inclusión y exclusión}
Se incluyen únicamente revisiones sistemáticas y metaanálisis publicados entre 2018 y 2025, en inglés o español, dado que la producción científica en el área comienza 
a incrementarse a partir de 2018, con la popularización de los modelos de lenguaje basados en transformers (p. ej., BERT, GPT-2, T5). Los estudios deben provenir de 
fuentes confiables los cuales son, a su vez, revisados por un experto. Se da preferencia a aquellos que presenten una cobertura de los temas más relevantes para 
el objeto de estudio. 
Se excluyen las revisiones narrativas, los documentos que carezcan de transparencia en sus métodos de búsqueda o síntesis, así como las publicaciones que no estén 
directamente relacionadas con el objeto de estudio delimitado.

\subsubsection{Identificación del estudio semilla y selección de revisiones relevantes}
El proceso de búsqueda se inicia con la identificación de dos estudios semilla, extraídos de Google Scholar con los parámetros de búsqueda "retrieval information" 
y "retrieval augmented generation", aplicando el filtro desde el año 2018 hasta 2025.

El primer estudio seleccionado fue Information Retrieval: Recent Advances and Beyond (Hambarde \& Proença, 2023), publicado en IEEE Access. 
Este trabajo constituye una revisión exhaustiva de la recuperación de información, abarcando desde los métodos tradicionales hasta los enfoques 
basados en deep learning y transformers, por lo que resulta un punto de partida principal para explorar la literatura reciente y relevante.

El segundo estudio semilla corresponde al artículo Retrieval-Augmented Generation for Large Language Models (Gao, Xiong, Gao, Jia, Pan, Bi, Dai, Sun \& Wang, 2023), publicado en
arXiv,
el cual presenta un marco conceptual y aplicado sobre la integración de recuperación de información y modelos generativos de gran escala. Su incorporación permite 
establecer una base teórica para contextualizar el análisis de las revisiones seleccionadas.

A partir de estos dos estudios semilla, y aplicando los criterios de inclusión y exclusión previamente definidos, se identificaron \textcolor{blue}{25} revisiones 
relevantes que cumplen con los criterios establecidos. 
Estas revisiones constituyen la base para el análisis y síntesis en el presente trabajo.


\subsubsection{Valoracion de la evidencias y extracion de la infomacion}
De los estudios seleccionados se procede a realizar un análisis, 
con el fin de excluir aquellos artículos que no cumplen con los criterios establecidos 
o que presentan un nivel de profundidad insuficiente para los objetivos de la revisión. 
La selección final de los estudios se realiza en consenso con expertos en el área, 
garantizando así la pertinencia y relevancia de la evidencia incluida.Para la organización, codificación y síntesis de la 
información se usa ATLAS.ti que facilitará la estructuración de los hallazgos.





% para resuemir la informacion se uso (herramienta) atlas por ejemplo

\subsubsection{Síntesis y representación de resultados}
Con la literatura seleccionada se identificó la hoja de ruta que se presenta a continuación en la figura \ref{fig:secciones-rag}.

% --- Grafico resumen

\begin{figure}[H]
\begin{center}
\begin{tikzpicture}[x=1cm,y=1cm]
  % distancia vertical entre pastillas
  \def\step{-1.5}

  % y inicial (arriba)
  \def\y{0}

  \pill{\y}{softcream}{I}{Fundamentos}
  \pgfmathsetmacro\y{\y+\step}
  \pill{\y}{softpink}{II}{Arquitectura}
  \pgfmathsetmacro\y{\y+\step}
  \pill{\y}{softblue}{III}{Fases de Implementación}
  \pgfmathsetmacro\y{\y+\step}
  \pill{\y}{softpeach}{IV}{Paradigmas}
  \pgfmathsetmacro\y{\y+\step}
  \pill{\y}{softyellow}{V}{Evaluación y metricas}
  \pgfmathsetmacro\y{\y+\step}
  \pill{\y}{softgreen}{VI}{Futuro de RAG}
  
\end{tikzpicture}
\end{center}
\caption{Resumen esquemático de RAG}
\label{fig:secciones-rag}
\end{figure}

A partir de esta hoja de ruta se desarrolla un esquema más detallado, en el que se expone primero exploraremos su teoría, 
características y aplicaciones, como se muestra en la Fig \ref{fig:Fudamentos}. Luego profundizamos en su arquitectura en la cual 
se describe cada uno de los componentes que lo conforman (retriever, augmented y generation) y las variantes y mejoras que existen de cada uno. 
Posteriormente, se detalla el proceso de implementación, desde la preparación de datos hasta el componente de generación, incluyendo las técnicas y herramientas más relevantes.
A continuación en la Fig , se examinan los paradigmas de RAG, dando a conocer los tipos de paradigmas y sus clases, para luego en la Fig , se presentan las métricas y evaluadores automáticos 
utilizados en la evaluación de sistemas RAG, así como las consideraciones éticas y de equidad que deben tenerse en cuenta.
Finalmente, se discuten las tendencias emergentes, los desafíos que actualmente se tienen y futuras direcciones que podrían tomar los sistemas RAG.



% --- Grafico fundamentos
\begin{figure}[H]
\begin{center}
\begin{tikzpicture}[x=1cm,y=1cm]
\tikzset{
  mindoval/.style={
    rounded corners=18pt,
    draw=black!70,
    line width=0.7pt,
    minimum width=3.2cm,
    minimum height=1.1cm,
    inner sep=6pt,
    fill=white      % <<< todos los nodos en blanco
  },
  mindcenter/.style={
    rounded corners=8pt,
    draw=black!70,
    line width=0.9pt,
    minimum width=3.6cm,
    minimum height=1.1cm,
    inner sep=4pt,
    fill=softrose,  % <<< centro en morado
    font=\bfseries
  },
  mindapp/.style={
    mindoval,
    fill=softblue   % <<< nodo aplicación en azul
  },
  mindarrow/.style={-latex, line width=0.9pt}
}

% --- Nodo central
\node[mindcenter] (fund) at (0,0) {Fundamentos};

% --- Nodos arriba
\node[mindoval] (aport)  at (0,3) {Aportes de RAGs};
\node[mindoval] (ragft)  at (5,0) {rag vs fine tuning};

% --- Nodos izquierda
\node[mindoval] (rolrag) at (-5,2.5) {rol de Rag};
\node[mindoval] (qrag)   at (-5,0.75) {¿Qué es RAG?};
\node[mindoval] (qllm)   at (-5,-1.0) {¿Qué es un LLM?};

% --- Nodos abajo
\node[mindapp, minimum width=3.0cm] (apli) at (0,-3.0) {Aplicación};
\node[mindoval] (texto)  at (-5,-4.5) {Texto};
\node[mindoval] (img)    at (-2,-6) {Imagen};
\node[mindoval, minimum width=3.6cm] (audio)  at (2,-6) {Audio y video};
\node[mindoval] (codigo) at (5,-4.5) {Código};

% --- Flechas (tocando bordes de nodos)
\draw[mindarrow] (fund.north) -- (aport.south);
\draw[mindarrow] (fund.east) -- (ragft.west);
\draw[mindarrow] (fund.west) -- (rolrag.east);
\draw[mindarrow] (fund.west) -- (qrag.east);
\draw[mindarrow] (fund.west) -- (qllm.east);
\draw[mindarrow] (fund.south) -- (apli.north);

\draw[mindarrow] (apli.west) -- (texto.east);
\draw[mindarrow] (apli) -- (img.north);
\draw[mindarrow] (apli) -- (audio.north);
\draw[mindarrow] (apli.east) -- (codigo.west);

\end{tikzpicture}
\end{center}
\caption{Fundamentos de RAG}
\label{fig:Fudamentos}
\end{figure}





\section{Enfoque Design Science Research (DSR)}
De acuerdo con vom Brocke et al. \cite{vomBrocke2020}, Design Science Research, desarrollada en 1969, es un paradigma de resolución de problemas que busca mejorar el conocimiento humano mediante la creación de artefactos innovadores. En otras palabras, es una metodología que crea soluciones a problemas reales y, al mismo tiempo, genera conocimiento útil y aplicable sobre cómo diseñar estas soluciones. Las etapas que se aplicarán en el presente trabajo son las siguientes:

\begin{itemize}[align=left, label=--]
\item \textbf{Identificación del problema y motivación} \
En esta etapa se precisa el problema y se justifica por qué es necesaria una solución. De acuerdo con Peffers et al. (2008), esta etapa exige analizar el problema en detalle, descomponiéndolo en sus partes clave para identificar sus causas, efectos y alcance. Además, es crucial justificar la relevancia del problema, tanto desde una perspectiva teórica (es decir, cómo contribuye al conocimiento académico) como desde una perspectiva práctica (cómo afecta a organizaciones, usuarios o sistemas reales).
También implica explorar la literatura para verificar que el problema es relevante, desafiante y nuevo, lo que permite definir los límites del proyecto de investigación.

\item \textbf{Definir los objetivos para la solución} \
Se plantean los criterios que debe cumplir una solución exitosa basándose en el conocimiento existente y en la factibilidad técnica y organizacional.
Los objetivos deberán permitir construir algo efectivo y deseable, no solamente desde el ámbito académico sino también en el entorno en que se aplicará. Estos pueden expresarse en términos cualitativos o cuantitativos; el investigador establece aquí la meta hacia donde se dirigirá el artefacto.

\item \textbf{Diseño y desarrollo del artefacto} \
En esta etapa se construye una solución concreta, como un modelo, software o sistema, que responde directamente a los objetivos planteados. Para ello, se utiliza el conocimiento existente que fundamenta las decisiones del diseño y la estructura del artefacto. No solo se trata de crear algo, sino de asegurar que pueda ser comprendido, evaluado y replicado por otros.

\item \textbf{Demostración del uso del artefacto para resolver el problema} \
Se muestra cómo se usa el artefacto en un escenario real o simulado. Esta demostración no valida científicamente su efectividad, sino que muestra su aplicabilidad, evidenciando que el artefacto propuesto puede operar de forma efectiva.
Por su parte, vom Brocke et al. (2020) destacan que esta etapa es fundamental para conectar el diseño teórico con la realidad del usuario o del entorno organizacional, permitiendo detectar oportunidades de mejora antes de una evaluación rigurosa.

\item \textbf{Evaluación del desempeño del artefacto} \
Se busca medir su efectividad, eficiencia e impacto, aportando evidencia que justifique su valor y utilidad. Además, según vom Brocke et al. (2020), esta puede asumirse de forma continua mediante una evaluación 
formativa que permita ciclos iterativos de rediseño y mejora a lo largo del proceso de investigación.
\item \textbf{Comunicación de los resultados al público académico y profesional} \
Finalmente, esta etapa consiste en difundir de forma clara los resultados del diseño y de la investigación realizada.
\end{itemize}

Estos pasos están basados en el modelo clásico de DSR de Peffers (2008), que vom Brocke adapta y expande en su guía.\\
\begin{landscape}
\begin{figure}[ht]
    \centering
    \resizebox{\linewidth}{!}{%
    \begin{tikzpicture}[scale=0.9, transform shape, node distance=1cm and 1.5cm,
        squarednode/.style={rectangle, draw=black, fill=white, very thick, minimum size=5mm, text width=4cm},
        everynode/.style={align=center}, 
        rectalbecirc/.style={
        draw=black,           % borde negro
        fill=white,           % fondo blanco
        line width=0.1pt,           % grosor del borde
        rounded corners=9pt,  % esquinas redondeadas
        align=center,         % alinear texto al centro
        text width=3cm,       % ancho del texto
        minimum height=1cm,   % altura mínima del nodo
        inner sep=6.5pt,
        font=\small
        }
        ]
        % nodos
        %horizontal
        \node[squarednode] (problema) {Identificación del problema};
        \node[squarednode] (objetivos) [right=of problema] {Objetivos para la solución};
        \node[squarednode] (diseno) [right=of objetivos] {Diseño y desarrollo del artefacto};
        \node[squarednode] (demostracion) [right=of diseno] {Demostración};
        \node[squarednode] (evaluacion) [right=of demostracion] {Evaluación del desempeño};
        \node[squarednode] (comunicacion) [right=of evaluacion] {Comunicación de resultados};
        %vertical
        \node[rectalbecirc] (start) [below=of problema] {Centinela tiene que presentar mejores resultados en las respuestas al momento de buscar información y generar contenido de valor para el usuario.};
        \node[rectalbecirc] (obj) [below=of objetivos] {
            \vspace*{-\baselineskip} % reduce el espacio inicial de la lista
            \begin{itemize} [left=0pt]\setlength\itemsep{1pt}\setlength\leftmargini{1pt}
                \renewcommand\labelitemi{\tiny$\bullet$}
                \item Mejorar la precisión de las respuestas.
                \item Aumentar la relevancia del contenido generado.
                \item Optimizar la experiencia del usuario.
            \end{itemize}
        };
        \node[rectalbecirc] (design) [below=of diseno] {Se contempla un diseño que integra retriver, augmented y generation lo que da como resultado un buscador el cual su respuesta es en lenguaje natural.};
        \node[rectalbecirc] (dem) [below=of demostracion] {Se demuestra el uso del buscador en un entorno controlado, mostrando su capacidad para responder preguntas y generar contenido relevante.};
        \node[rectalbecirc] (eval) [below=of evaluacion] {Se evalúa el buscador mediante métricas estándar y FATE};
        \node[rectalbecirc] (com) [below=of comunicacion] {En el presente trabajo se presenta el proceso que se ha realizado para el desarrollo del buscador, así como los resultados obtenidos y recomendaciones para futuras mejoras.};

        %conexiones horizontales
        \draw[-to] (problema) -- (objetivos);
        \draw[-to] (objetivos) -- (diseno);
        \draw[-to] (diseno) -- (demostracion);
        \draw[-to] (demostracion) -- (evaluacion);
        \draw[-to] (evaluacion) -- (comunicacion);
        %conexiones verticales
        \draw[-to] (problema) -- (start);
        \draw[-to] (objetivos) -- (obj);
        \draw[-to] (diseno) -- (design);    
        \draw[-to] (demostracion) -- (dem);
        \draw[-to] (evaluacion) -- (eval);
        \draw[-to] (comunicacion) -- (com);

        %contenedor
        \node[draw=black, rounded corners=10pt, thick, minimum width=23cm, minimum height=4cm, fill=gray!5, anchor=north, name=contenedor] 
at ([yshift=-0.75cm]current bounding box.south) {};
        \coordinate (center) at (contenedor.center);
        \def\sep{7} % Separación horizontal
        % Elipses dentro del contenedor
        \node[draw=black, ellipse, minimum width=3cm, minimum height=2.2cm, fill=blue!10, align=center, font=\small] 
            (elipse1) at ($(center) + (-\sep,0)$) 
            {Definición de RAG};

        \node[draw=black, ellipse, minimum width=3cm, minimum height=2.2cm, fill=blue!10, align=center, font=\small] 
            (elipse2) at ($(center)$) 
            {Arquitectura de RAG};

        \node[draw=black, ellipse, minimum width=3cm, minimum height=2.2cm, fill=blue!10, align=center, font=\small] 
            (elipse3) at ($(center) + (\sep,0)$) 
            {Optimización de RAG};
        \node[align=center, text width=30cm, yshift=-1.5cm] at (center) {\fontsize{10}{12}\selectfont
    Puntos de entrada a la investigación};

        % Conexiones entre elipses y nodos
        \draw[-to] (elipse1) -- (start);
        \draw[-to] (elipse1) -- (obj);
        \draw[-to] (elipse2) -- (design);
        \draw[-to] (elipse3) -- (eval);
        \draw[-to] (elipse3) -- (dem);
        \draw[-to, bend right=20] (comunicacion) to node[pos=0.06, above,sloped]{Iterativo} (objetivos);
        \draw[-to, bend right=20] (comunicacion) to (diseno);
        \draw[-to, bend right=20] (comunicacion) to (evaluacion);
    \end{tikzpicture}
    }
    \caption{Proceso de Diseño de Investigación para el desarrollo de RAG en Centinela}
\end{figure}
\end{landscape}

\section{Diseño y desarrollo
del artefacto}