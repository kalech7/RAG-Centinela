% -- Formato para ejemplos
\newtheoremstyle{estiloejemplo}%
    {9pt}{9pt}%
    {}%
    {0pt}%
    {\bfseries\scshape}{.}%
    { }%
    {}

% -- Ambiente para ejemplos
\theoremstyle{estiloejemplo}
    \newtheorem{ejemplo}{Ejemplo}
    \newtheorem*{observacion}{Observación}

% -- Formato para teoremas
\newtheoremstyle{estiloteorema}%
    {9pt}{9pt}%
    {\slshape}%
    {0pt}%
    {\bfseries\scshape}{.}%
    { }%
    {}

% -- Ambiente para teoremas
\theoremstyle{estiloteorema}
    \newtheorem{definicion}{Definición}[chapter]
    \newtheorem{axioma}{Axioma}
    \newtheorem{teorema}{Teorema}[chapter]
    \newtheorem{corolario}[teorema]{Corolario}
    \newtheorem{proposicion}[teorema]{Proposición}
    \newtheorem{lema}[teorema]{Lema}

% -- Función con 5 argumentos
\newcommand{\funcion}[5]{%
    {\setlength{\arraycolsep}{2pt}
        \begin{array}{r@{}ccl}
        #1\colon & #2 & \longrightarrow & #3\\
        & #4 & \longmapsto & #5
        \end{array}}
    }

% -- Función con 3 argumentos
\newcommand{\func}[3]{%
    #1\colon  #2  \to  #3
}

% -- Clausura
\newcommand{\cl}[1]{\overline{#1}}

% -- Conjuntos
\newcommand{\R}{\mathbb{R}}
\newcommand{\N}{\omega}

% -- Sucesiones
\newcommand{\suc}[2][n]{
    \left(#2\right)_{#1\in\naturales}
}

% -- Comentario
\newcommand{\comentario}[1]{
    {\color{red!50!black}\textsf{#1}}}
    
\newenvironment{Comentario}{
    \bgroup
    \color{red!50!black}
    \sffamily
    }
    {
    \egroup
    }

% -- Estilo de listas
\setlist[itemize,1]{label=$\bullet$}

% -- Permitir cortes de ecuaciones
\allowdisplaybreaks 